\section{Conclusions}

%The task of automatic cell boundary segmentation is difficult, and trying to improve such segmentations automatically as a post-process through merge and split error correction is, in principle, no different than trying to improve the underlying cell boundary segmentation. Due to the task difficulty, manual proofreading of connectomics segmentations is necessary, but it is a time consuming and error-prone task. Humans are the bottleneck and minimizing the manual labor is the goal.
%We have trained classifiers to find merge and split errors in automatic segmentations of connectomics data. We then reduce the correction of these errors to simple yes/no decisions. This allows proofreading with better performance than existing systems.
%Our experiments also show that automatic proofreading works. This has the potential to further reduce the manual labor and will be the target of future research. To encourage testing of our proposed architecture and replication of our experiments, we provide our framework and data as free and open research at (link omitted for review).
%Minimizing manual labor is the goal when proofreading large automatic segmentations. Our classifiers suggest potential errors and corrections better than existing systems. This is especially important for novice users. Novices report difficulties when using fully interactive software but are invaluable to proofread the shear amounts of connectomics data.
%Our experiments also show that automatic proofreading works. This has the potential to further reduce human involvement and will be the target of future research. We provide our framework and data as free and open research at (link omitted for review).
Humans are the bottleneck when proofreading segmentation data and minimizing the manual labor is the goal. Our classifiers suggest potential errors and corrections better than existing methods. This reduces the time spent finding and correcting errors.
Our experiments also show that automatic proofreading has potential to further reduce human involvement. This will be the target of future research. We provide our framework and data as free and open research at \url{http://rhoana.org/guidedproofreading/}.

\section*{Acknowledgements}
We would like to thank Stephen Plaza for detailed explanations of focused proofreading and Toufiq Parag for the configuration of the NeuroProof classifier.

