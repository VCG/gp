\section{Conclusions}

%The task of automatic cell boundary segmentation is difficult, and trying to improve such segmentations automatically as a post-process through merge and split error correction is, in principle, no different than trying to improve the underlying cell boundary segmentation. Due to the task difficulty, manual proofreading of connectomics segmentations is necessary, but it is a time consuming and error-prone task. Humans are the bottleneck and minimizing the manual labor is the goal.
%We have addressed this problem through training a convolutional neural network to detect ambiguous regions from labeled data---in effect, by finding a non-linear mapping between image and segmentation data. This allows us to identify merge and split errors with better performance than existing systems. Our experiments have shown that guided proofreading has the potential to reduce the bottleneck in the analysis of large connectomics datasets. To encourage testing of our proposed architecture and replicate our experiments, we provide our framework and data as free and open research at (link omitted for review).