\documentclass[10pt,twocolumn,letterpaper]{article}

%\usepackage{cvpr}
\usepackage{iccv}
\usepackage{microtype}
\usepackage{times}
\usepackage{epsfig}
\usepackage{graphicx}
\usepackage{amsmath}
\usepackage{amssymb}
\usepackage{caption}
\usepackage{subcaption}
\usepackage{booktabs}
\usepackage{makecell}
\usepackage{authblk}
% Include other packages here, before hyperref.

% If you comment hyperref and then uncomment it, you should delete
% egpaper.aux before re-running latex.  (Or just hit 'q' on the first latex
% run, let it finish, and you should be clear).
\usepackage[pagebackref=true,breaklinks=true,letterpaper=true,colorlinks,linkcolor=blue,citecolor=blue,bookmarks=false]{hyperref}

\newcommand{\JT}[1]{\textcolor{blue}{JT: #1}}
\newcommand{\HP}[1]{\textcolor{red}{HP: #1}}
\newcommand{\DFH}[1]{\textcolor{red}{DH: #1}}

\iccvfinalcopy % *** Uncomment this line for the final submission

%\def\cvprPaperID{1234} % *** Enter the CVPR Paper ID here
\def\iccvPaperID{1234}
\def\httilde{\mbox{\tt\raisebox{-.5ex}{\symbol{126}}}}

% Pages are numbered in submission mode, and unnumbered in camera-ready
%\ifcvprfinal\pagestyle{empty}\fi
\ificcvfinal\pagestyle{empty}\fi
\begin{document}

%%%%%%%%% TITLE
\title{Guided Proofreading of Automatic Segmentations for Connectomics}

\author[1,2]{Daniel Haehn\thanks{Corresponding author,  \url{haehn@seas.harvard.edu}}}
\author[1,2]{Verena Kaynig}
\author[3]{James Tompkin}
\author[2]{Jeff W. Lichtman}
\author[1,2]{Hanspeter Pfister}
\affil[1]{Harvard Paulson School of Engineering and Applied Sciences, Cambridge, MA 02138, USA}
\affil[2]{Harvard Brain Science Center, Cambridge, MA 02138, USA}
\affil[3]{Brown University, Providence, RI 02912, USA}


\maketitle
%\thispagestyle{empty}

%%%%%%%%% ABSTRACT
\begin{abstract}
%
Automatic cell image segmentation methods in connectomics produce merge and
split errors, which require correction through proofreading. Previous research
has identified the visual search for these errors as the bottleneck in
interactive proofreading. To aid error correction, we develop two classifiers
that automatically recommend candidate merges and splits to the user. These
classifiers use a convolutional neural network (CNN) that has been trained with
errors in automatic segmentations against expert-labeled ground truth. Our
classifiers detect potentially-erroneous regions by considering a large context
region around a segmentation boundary. Corrections can then be performed by a
user with yes/no decisions, which reduces variation of information $7.5\times$ faster than previous
proofreading methods. We also present a fully-automatic mode that uses a
probability threshold to make merge/split decisions. Extensive experiments using
the automatic approach and comparing performance of novice and expert users
demonstrate that our method performs favorably against state-of-the-art
proofreading methods on different connectomics datasets.
%
\end{abstract}

%%%%%%%%% BODY TEXT
\section{Introduction}

In connectomics, neuroscientists annotate neurons and their connectivity within 3D volumes to gain insight into the functional structure of the brain. Rapid progress in automatic sample preparation and electron microscopy (EM) acquisition techniques has made it possible to image large volumes of brain tissue at $\approx4\, nm$ per pixel to identify cells, synapses, and vesicles. For $40\, nm$ thick sections, a $1\, mm^3$ volume of brain contains $10^{15}$ voxels, or 1 petabyte of data. With so much data, manual annotation is infeasible, and automatic annotation methods are needed~\cite{jain2010,Liu2014,GALA2014,kaynig2015large}.

%Automatic annotation by segmentation and classification of brain tissue is challenging~\cite{isbi_challenge}. The state of the art uses supervised learning with convolutional neural networks~\cite{Ciresan:2012f}, or potentially even unsupervised learning~\cite{BogovicHJ13}. Typically, cell membranes are detected in 2D images, and the resulting region segmentation is grouped into geometrically-consistent cells across registered sections. Cells may also be segmented across registered sections in 3D directly. Using dynamic programming techniques~\cite{Masci:2013a} and a GPU cluster, these classifiers can segment $\approx1$ terabyte of data per hour ~\cite{kasthuri2015saturated}. This is sufficient to keep up with the 2D data capture process on state-of-the-art electron microscopes (though 3D registration is still an expensive offline operation).

Automatic annotation by segmentation and classification of brain tissue is challenging~\cite{isbi_challenge} and all the methods make errors. This means we are left with large data which needs \emph{proofreading} by humans. This crucial task serves two purposes: 1) to correct errors in the segmentation, and 2) to provide a large body of labeled data to train better automatic segmentation methods. Recent proofreading tools provide intuitive user interfaces to browse segmentation data in 2D and 3D and to identify and manually correct errors ~\cite{markus_proofreading,raveler,mojo2,haehn_dojo_2014}. Many kinds of errors exist, such as inaccurate boundaries, but the most common are \emph{split errors}, where a single segment is labeled as two, and \emph{merge errors}, where two segments are labeled as one (Fig.~\ref{fig:merge_and_slit_errors}). With user interaction, split errors can be joined, and the missing boundary in a merge error can be defined with manually-seeded watersheds~\cite{haehn_dojo_2014}. However, even with semi-automatic correction tools, the visual inspection to find errors takes the majority of the time~\cite{proofreading_bottleneck}.

\begin{figure}[t]
\begin{center}
  \includegraphics[width=\linewidth]{gfx/merge_and_split_errors.png}
\end{center}
\vspace{-4mm}
   \caption{The most common proofreading corrections are fixing split errors (red arrows) and merge errors (yellow arrow). A fixed segmentation matches the cell borders.}
\label{fig:merge_and_slit_errors}
\end{figure}

Our goal is to add automatic detection of split and merge errors to proofreading tools. We design automatic classifiers that detect split and merge errors in segmentations so the user does not need to visually inspect the whole data volume to spot errors. A proofreading tool then recommends regions with a high probability of an error to the user, and suggest corrections to accept or reject. We call this process \textit{guided proofreading}.

In this paper, we introduce classifiers to detect merge- and split errors based on a convolutional neural network (CNN). We believe that this is the first time that deep learning is applied to the task of proofreading. Our classifiers work on top of any existing automatic segmentation method to find potential errors and suggest corrections. Given a membrane segmentation from a fast automatic method, our classifiers operate on the boundaries of whole cell regions. Compared to techniques that must analyze every input pixel, we reduce the data analysis to the boundaries. First, we train a CNN to detect only split errors. The output of this network is a probability whether a boundary between two segments is valid or not. We then reuse the same network to also detect merge errors by generating possible boundaries within a cell and inverting the split error score. We create corrections for both types of errors which can be accepted or rejected. This reduces the proofreading operation to simple yes/no decisions.

We further propose a greedy algorithm to perform proofreading. Possible erroneous regions are sorted by their score and the algorithm iteratively suggests a correction for each region. A user then works through this stream of regions and corrections. In a forced choice setting, the user either selects a correction or skips it to advance to the next region. This choice can be also performed automatically by running the algorithm until a configurable threshold is reached. In addition, if ground truth data is available, we can use a selection oracle to drive the forced choice selection. The oracle only accepts corrections which improve the automatic segmentation. This equals perfect proofreading.

We evaluate our method automatically by threshold and oracle on multiple real-world connectomics datasets. To evaluate the forced choice setting, we perform a quantitative user study. The study targets non-experts with no previous experience of proofreading electron microscopy data. We ask the participants to proofread a small segmentation volume in a fixed time frame by performing yes/no decisions. The user study is designed as a between-subjects experiment and compares guided proofreading against two other methods: a recently published fully interactive proofreading tool named \textit{Dojo} by Haehn~\etal~\cite{haehn_dojo_2014} and the semi-automatic \textit{focused proofreading} approach by Plaza~\cite{focused_proofreading}. We also asked four domain experts to use guided proofreading and focused proofreading for additional comparison.

Our first contribution is a classifier for split error detection based on a convolutional neural network. The classifier performs well even when trained with little amounts of training data. This is important since generating ground truth labels in connectomics requires manually labeling pixels and is very time-consuming. Our second contribution is a mechanism to identify merge-errors by re-using the split error classifier. Merge errors are usually less common than split errors in the oversegmented automatic labelings. However, they require more interaction during correction since split lines need to be manually drawn. Our method reduces this to a single click by providing the potential correction. The split and merge error identification is executed as a greedy algorithm to correct segmentation volumes, the third contribution of this paper. The algorithm can be driven automatically with a threshold, by an oracle based on ground truth and interactively in a forced choice setting. Our final contribution is our quantitative user study. We present statistically significant results showing that novice and expert users of guided proofreading are able to proofread a given dataset better and faster than with existing interactive and semi-automatic proofreading tools. As a consequence, we are able to provide tools to proofread segmentations more efficiently, and so better tackle large volumes of connectomics imagery.



%The reason why we work on top of an existing segmentation rather than improving the underlying technique, is to explore the error space in a non-linear way. The real world analogy would be inspecting something with a second set of eyes and different expertise.


\begin{figure*}[t]
\begin{center}
\includegraphics[width=\linewidth]{gfx/architecture.png}
\end{center}
  \vspace{-4mm}
   \caption{We build the guided proofreading classifiers using a traditional CNN architecture. The network is based on four convolutional layers, each followed by max pooling as well as dropout regularization. The 4-channel input patches are rated as either correct splits or as split errors.}
\label{fig:architecture}
\end{figure*}

\section{Related Work}

\textbf{Automatic Segmentation.} Multi-terabyte EM brain volumes require automatic segmentation~\cite{jain2010,Liu2014,NunezIglesias2013Machine,GALA2014}, but can be hard to classify due to ambiguous intercellular space: the 2013 IEEE ISBI neurites 3D segmentation challenge~\cite{isbi_challenge} showed that existing algorithms which learn from expert-segmented training data still exhibit high error rates.

NeuroProof \cite{neuroproof2013} tries to decrease error rates with interactive learning of agglomeration of over-segmentations of images, based on a random forest classifier. Vazquez-Reina \etal~\cite{amelio_segmentation} propose automatic 3D segmentation by taking whole EM volumes into account rather than a per section approach, then solving a fusion problem with a global context. Kaynig \etal~\cite{kaynig10} propose a random forest classifier coupled with an anisotropic smoothing prior in a conditional random field framework with 3D segment fusion. It is also possible to learn segmentation classification features directly from images with CNNs. Ronneberger \etal~\cite{RonnebergerFB15} use a contracting/expanding CNN path architecture to enable precise boundary localization with small amounts of training data. Lee \etal~\cite{lee2015recursive} recursively train very deep networks with 2D and 3D filters to detect boundaries. 

Bogovic \etal~\cite{BogovicHJ13} learn 3D features, and show even that unsupervised learning can produce better features than hand-designs. Our work was inspired by this paper and we extend the features reported by Bogovic \etal for our guided proofreading classifiers as described in section \ref{sec:methods}.
These approaches make good progress; however, in general, proofreading is required to correct errors. 

\textbf{Interactive Proofreading.} While proofreading is very time consuming, it is fairly easy for humans to perform corrections through splitting and merging segments. One way to perform such corrections is by using expert tools such as Raveler introduced by Chklovskii~\etal~\cite{chklovskii2010, raveler}. This software offers many parameters for tweaking the proofreading process. Created in 2010, Raveler is still used today by professional full-time proofreaders. Many similar systems exist as stand-alone products or plugins to existing visualization system,~\eg V3D~\cite{proofreading_bottleneck} or AVIZO~\cite{markus_proofreading}. 

In contrast to these expert tools, recent works attack the problem of proofreading massive datasets by novices through crowd-sourcing~\cite{saalfeld09,anderson2011,Giuly2013DP2}. A very popular platform is EyeWire presented by Kim \etal~\cite{eyewire_nature}. EyeWire is set up as an online game and participants earn virtual rewards for merging oversegmented labeling to reconstruct the retina cells.
A range of proofreading tools exist in-between expert systems and online games such as Mojo and \textit{Dojo} developed by Haehn \etal~\cite{haehn_dojo_2014,Neuroblocks}. Mojo provides a simple scribble interface for error correction, and Dojo extends this for distributed proofreading via a minimalistic web-based user interface. The authors define requirements for general proofreading tools, and then evaluate the accuracy and speed of Raveler, Mojo, and Dojo through a quantitative user study (Sec. 3 and 4)~\cite{haehn_dojo_2014}. In this paper, we use the Dojo system as a baseline for interactive proofreading and extend the experiment reported by Haehn \etal, where Raveler, Mojo, and Dojo are compared in terms of accuracy and speed.
All interactive proofreading solutions require the user to find potential errors manually which takes the majority of time~\cite{proofreading_bottleneck,haehn_dojo_2014}. Recent works propose computer-aided proofreading systems which help with this visual search task.

%Most relevant methods here use heuristics to analyze the image data to find potential errors. 
\textbf{Computer-aided Proofreading.} To reduce the time spent looking for errors, Plaza proposed \textit{focused proofreading} (FP) ~\cite{focused_proofreading}. His approach finds split errors by analyzing segment size ratios across slices and then offers yes/no questions to correct these errors. Plaza reports that additional processing beyond FP is required to find merge errors. His method is freely available as open source software and is integrated into Raveler. This makes it feasible for us to use FP as a baseline for evaluating guided proofreading as described in section \ref{sec:evaluation}.

A similar approach was published by Karimov \etal as guided volume editing~\cite{karimov_guided_volume_editing}. Measuring differences in histogram distributions in image data enables to find potential split and merge errors in the corresponding segmentation. For merge errors, the authors generate possible boundaries using watershed which inspired our approach as described in section \ref{sec:methods}. Guided volume editing was designed to let expert users correct labeled computer-tomography datasets by performing several interactions per correction. 

While focused proofreading and guided volume editing both use a heuristical approach to analyze the image data, Uzunbas \etal showed that potential labeling errors can be found by considering the merge tree of an automatic segmentation method~\cite{uzunbas}. The authors track uncertainty throughout the automatic labeling by training a conditional random field. This method is really a segmentation technique but it is possible to use the uncertainty information to present potential regions for proofreading. Their method requires further work to overcome the requirement of isotropic volumes, a property not given for most connectomics datasets. Our approach, guided proofreading, works on isotropic as well as anisotropic data, and finds merge and split errors.



\section{Method}
\label{sec:methods}

%We first describe our classifier for detecting split errors which is based on a convolutional neural network (CNN). We detail the CNN architecture, input features and the training method. We then describe how the same classifier can be used to detect merge errors and how we create potential corrections. The classifiers are integrated into an existing proofreading workflow as reported after. Finally, we explore an active label suggestion method which reorders the ranking obtained by our classifiers and maximizes the information gain provided by each potential correction.
\subsection{Split Error Detection}
\label{sec:spliterrordetection}

We build a split error classifier with output $p$ using a CNN to check whether an edge within an existing automatic segmentation is valid ($p=0$) or not ($p=1$). Rather than analyzing every input pixel, the classifier operates only on segment boundaries, which requires less pixel context and is faster. In contrast to Bogovic \etal~\cite{BogovicHJ13}, we work with 2D slices rather than 3D volumes. This enables proofreading prior or in parallel to a computationally expensive stitching and 3D alignment of individual EM images.

\paragraph{CNN Architecture.} Boundary split error detection is a binary classification task since the boundary is either correct or erroneous. However, in reality, the score $p$ is between 0 and 1. In connectomics, classification complexity arises from hundreds of different cell types, rather than from the classification decision itself. Intuitively, this yields a wider architecture with more filters rather than a deeper architecture with more layers. We explored different architectures---including residual networks~\cite{resnet}---with brute force parameter searches and precision and recall comparisons (see supplementary materials). Our final CNN configuration for split error detection has four convolutional layers, each followed by max pooling with dropout regularization to prevent overfitting due to limited training data (Fig.~\ref{fig:architecture}).

\paragraph{Classifier Inputs.} To train the CNN, we consider boundary context in the decision making process via a $75\times75$ patch over the center of an existing boundary. This size covers approximately $80\%$ of all boundaries in the 6~nm Mouse S1 AC3 Open Connectome Project dataset. If the boundary is not fully covered, we sample up to 10 non-overlapping patches along the boundary, and average the resulting scores weighted by the boundary length coverage per patch.
%\footnote{\scriptsize{\url{http://openconnectomeproject.org/}}}

Similar to Bogovic~\etal~\cite{BogovicHJ13}, we use grayscale image data, corresponding boundary probabilities, and a single binary mask combining the two neighboring labels as inputs to our CNN. However, we observed that the boundary probability information generated from EM images is often misleading due to noise or artifacts in the data. This can result in merge errors within the automatic segmentation. To better direct our classifier to train on the true boundary, we extract the border between two segments. Then, we dilate this border by 5 pixels to consider slight edge ambiguities and use this binary mask as an additional input. This creates a stacked 4-channel input patch. Fig.~\ref{fig:cnn_inputs} shows examples of correct and erroneous input patches and their corresponding automatic segmentation and ground truth.

\begin{figure}[t]
\centering
\includegraphics[width=\linewidth]{gfx/cnn_inputs.pdf}
\caption{Example inputs for learning correct splits and split errors (candidate segmentation versus the ground truth). Image, membrane probabilities, merged binary labels, and a dilated border mask provide 4-channel input patches.}
\label{fig:cnn_inputs}
\end{figure}


\subsection{Merge Error Detection}

Identification and correction of merge errors is more challenging than finding and fixing split errors, because we must look inside segmentation regions for missing or incomplete boundaries and then propose the correct boundary. However, we can reuse the same trained CNN for this task. Similar to guided volume editing by Karimov~\etal~\cite{karimov_guided_volume_editing}, we generate potential borders within a segment. For each segmentation label, we dilate the label by 20 pixel and generate 50 potential boundaries through the region by randomly placing watershed seed points at opposite sides of the label boundary. We perform watershed on the inverted grayscale EM image. This yields 50 candidate splits.

Dilation of the segment prior to watershed is motivated by our observation that the generated splits tend to attach to real membrane boundaries. These boundaries are then individually rated using our split error classifier. For this, we invert the probability score such that a correct split (previously encoded as $p=0$) is most likely a candidate for a merge error (now encoded as $p=1$). In other words, if a generated boundary is ranked as correct, it probably should be in the segmentation. Fig. \ref{fig:merge_error} illustrates this procedure.

\begin{figure}[t]
\centering
\includegraphics[width=\linewidth]{gfx/merge_error_v6.pdf}
\caption{Merge error detection: Potential borders are generated on inverted images by randomly placing watershed seeds (green) on the boundary of a dilated segment. The best ranked seeds and border (both in red) result in the shown error correction.}
%Merge errors are identified by generating randomly-seeded watershed borders within a dilated label segment. Then, each border is individually rated using the split error CNN by inverting the probability score. A confident rating for a correct split most likely indicates the missing border of the merge error, and can be used to correct the labeling.}
\label{fig:merge_error}
\end{figure}

\subsection{Error Correction}
\label{sec:errorcorrection}

We combine the proposed classifiers to perform corrections of split and merge errors in automatic segmentations. For this, we first perform merge error detection for all existing segments in a dataset and store the inverted rankings $1-p$ as well as potential corrections. After that, we perform split error detection and store the ranking $p$ for all neighboring segments in the segmentation. Then, we sort the merge and split error rankings separately from highest to lowest. For error correction, first we loop through the potential merge error regions and then through the potential split error regions. During this process, each error is now subject to a yes/no decision which can be provided in different ways:

\paragraph{Selection oracle.} If ground truth data is available, the selection oracle \textit{knows} whether a possible correction improves an automatic segmentation. This is realized by simply comparing the outcome of a correction using a defined measure. The oracle only accepts corrections which improve the automatic segmentation---others get
discarded. This is guided proofreading with a perfect user, and allows us to assess the upper limit of improvements.

\paragraph{Automatic selection with threshold.} The decision whether to accept or reject a potential correction is taken by comparing rankings to a threshold $p_t$. If the inverted score $1-p$ of a merge error is higher than a threshold $1-p_t$, the correction is accepted. Similarly, a correction is accepted for a split error if the ranking $p$ is higher than $p_t$. Our experiments have shown that the threshold $p_t$ is the same for merge and split errors for a balanced classifier that has been trained on equal numbers of correct and error patches.

\paragraph{Forced choice setting.} We present a user with the choice to accept or reject a correction. All potential split errors are seen. Inspecting all merge errors is not possible for users due to the sheer amount of generated borders. Therefore, we only present merge errors that have a probability threshold higher than $1-p_t$.

\noindent \newline In all cases, a decision has to be made to advance to the next possible erroneous region. If a merge error correction was accepted, the newly found boundary is added to the segmentation data. This partially updates the merge error and split error ranking with respect to the new segment. If a split error correction was accepted, two segments are merged in the segmentation data and the disappearing segment is removed from all error rankings. Then, we perform merge error detection on the now larger segment and update the ranking. We also update the split error rankings to include all new neighbors, and re-sort. The error with the next highest ranking then forces a choice.

\subsection{User Interface}

We integrate guided proofreading into an existing large data connectomics workflow. The web-based system is designed with a novice-friendly user interface (Fig.~\ref{fig:ui}). We show the current labeling of a cell boundary outline and its proposed correction overlayed on EM image data. The user cannot distinguish the current labeling from the proposed correction to avoid selection bias. We also show a solid overlay of the current and the proposed labeling. In addition, we show the image without overlays to provide an unoccluded view. User interaction is simple and involves one mouse click on either the current labeling or the correction. After interaction, the next potential error is shown.

\begin{figure}[t]
\includegraphics[width=\linewidth]{gfx/user_interface_split.pdf}
\caption{User interface. A candidate error region is shown on the left. The user must choose between the region being a split error which needs correcting (center) or not (right). Confirming the choice advances to the next potential error.} % Hovering highlights the current selection with a blue border, and a}
\label{fig:ui}
\end{figure}

%
%\subsection{Active Label Suggestion}
%
%In an interactive setting, one way to present patches to the user for proofreading is to order them by the confidence probability of the GP classifier. However, in an active learning setting, where the network is retrained repeatedly on new label evidence, this approach is less likely to decrease segmentation error as, with the new labels, we are only reinforcing what the network already has a high confidence in.
%Instead, we apply active label suggestion to guide the user into labeling patches which will be more informative to retraining, and so overall decrease VI faster within the proofreading cycle of label $\rightarrow$ train $\rightarrow$ label. For each patch, we remove the softmax classification layer and look at the activation weights associated with the last dense layer. These become a high-dimensional feature vector. Then, we adapt Anon~\etal~\cite{ANON} to provide label suggestions based on features from the learned CNN, which is based on maximizing the average information gain provided by a candidate patch to label.
%A second consideration is that each patch labeled by the user provides evidence to other patches, e.g., correcting a split error redefines an entire boundary, from which multiple candidate patch labelings could have been drawn. As such, when the user labels a patch, we consider all `knock-on' effect patches as also being labeled, and feed these into the active label suggestion system similarly.
%In section \ref{sec:evaluation}, we report the difference in performance from using active label suggestion rather than confidence ordering when presenting patches to the user. These results are without retraining the network after new labelings: this should improve results, but would have to be batched to reduce computational load; hence, we leave this for future work.


\section{Evaluation}
\label{sec:evaluation}

We evaluate guided proofreading on multiple different real-world connectomics datasets of different species. All datasets were acquired using either serial section electron microscopy (ssEM) or serial section transmission electron microscopy (ssTEM). We perform experiments with the selection oracle, with automatic selection with threshold, and in the forced choice setting via a between-subjects user study with both novice and expert participants.

\subsection{Datasets}

\paragraph{L. Cylinder.} We use the left part of the 3-cylinder mouse cortex volume of Kasthuri \etal~\cite{kasthuri2015saturated} ($2048\times2048\times300$ voxels). The tissue is dense mammalian neuropil from layers 4 and 5 of the S1 primary somatosensory cortex, acquired using ssEM. The dataset resolution is $3\times3\times30~\text{nm}^3\text{/voxel}$. Image data and a manually-labeled expert `ground truth' segmentation is publicly available\footnote{\scriptsize{\url{https://software.rc.fas.harvard.edu/lichtman/vast/}}}.

\paragraph{AC4 subvolume.} This is part of a publicly-available dataset of mouse cortex that was published for the ISBI 2013 challenge ``SNEMI3D: 3D Segmentation of neurites in EM images''. The dataset resolution is $6\times6\times30~\text{nm}^3\text{/voxel}$ and it was acquired using ssEM. Haehn~\etal~\cite{haehn_dojo_2014} found the most representative subvolume ($400\times400\times10$ voxels) of this dataset with respect to the distribution of object sizes, and used it for their interactive connectomics proofreading tool experiments. We use their publicly available data, labeled ground truth, and study findings\footnote{\scriptsize{\url{http://rhoana.org/dojo/}}}.

\paragraph{Automatic segmentation pipeline.}
We use a state-of-the-art method to create a dense automatic segmentation of the data. Membrane probabilities are generated using a CNN based on the U-net architecture (trained exclusively on different data than the GP classifiers)~\cite{RonnebergerFB15}. The probabilities are used to seed watershed and generate an oversegmentation using superpixels. Agglomeration is then performed by the GALA active learning classifier with a fixed agglomeration threshold of 0.3~\cite{nunez2014graph}. We describe this approach in the supplemental material.

\subsection{Classifier Training}

We train our split error classifier on the L. Cylinder dataset. We use the first 250 sections of the data for training and validation. For n-fold cross validation, we select one quarter of this data and re-select after each epoch. We minimize cross-entropy loss and update using stochastic gradient descent with Nesterov momentum~\cite{nesterov}. To generate training data, we identify correct regions and split errors in the automatic segmentation by intersection with ground truth regions. This is required since extracellular space is not labeled in the ground truth, but is in our dense automatic segmentation. From these regions, we sample 112,760 correct and 112,760 split error patches with 4-channels (Sec.~\ref{sec:spliterrordetection}). The patches are normalized. To augment our training data, we rotate patches within each mini-batch by $k*90$ degrees with randomly chosen integer $k$. The training parameters such as filter size, number of filters, learning rate, and momentum are the result of intuition and experience, studying recent machine learning research, and a limited brute force parameter search (see supplementary material). 

Table~\ref{tab:parameters} lists the final parameters. Our CNN configuration results in 171,474 learnable parameters. We assume that training has converged if the validation loss does not decrease for 50 epochs. We test the CNN by generating a balanced set of 8,780 correct and 8,780 error patches using unseen data of the left cylinder dataset. 

\begin{table}[t]
\caption{Training parameters, cost, and results of our guided proofreading classifier versus focused proofreading by Plaza~\cite{focused_proofreading}. Both methods were trained on the same mouse brain dataset using the same hardware (Tesla X GPU).}
\small{
\begin{tabular}{ll}
	\toprule
	\begin{tabular}{l}
		\textbf{Guided Proofreading} \\ \midrule
		\emph{Parameters} \\ \midrule
		Filter size: 3x3 \\ No. Filters 1: 64 \\ No. Filters 2--4: 48 \\ Dense units: 512 \\ Learning rate: 0.03--0.00001\\ Momentum: 0.9--0.999\\Mini-Batchsize: 128 \\
	\end{tabular}
	&
	\begin{tabular}{l}
		\vspace{0.2mm} \\
		\midrule
		\emph{Results---Test Set} \\ \midrule Cost [m]: 383 \\ Val. loss: 0.0845 \\ Val. acc.: 0.969 \\ Test. acc.: 0.94 \\ Prec./Recall: 0.94/0.94 \\ F1 Score: 0.94 \\ ~ \\
	\end{tabular}
\end{tabular}

\vspace{0.5mm}
\begin{tabular}{ll}
	\toprule
	\begin{tabular}{l}
		\textbf{Focused Proofreading}\\ \midrule
		\emph{Parameters} \\ \midrule
		Iterations: 3 \\
		Learning strategy: 2\\
		Mito agglomeration: Off~~~~~~ \\  
		Threshold: 0.0\\~\\
	\end{tabular}
	&
	\begin{tabular}{l}
		\vspace{0.2mm} \\
		\midrule
		\emph{Results---Test Set} \\ \midrule Cost [m]: 217 \\ Val. acc.: 0.99 \\ Test. acc.: 0.68 \\ Prec./Recall: 0.58/0.56 \\ F1 Score: 0.54 \\
	\end{tabular}
%	\bottomrule
\end{tabular}
\hrule
}
\label{tab:parameters}
\end{table}

\subsection{Baseline Comparisons}

\paragraph{Interactive proofreading.} Haehn~\etal's comparison of interactive proofreading tools concludes that novices perform best when using Dojo~\cite{haehn_dojo_2014}. We studied the publicly available findings of their user study and use the data of all Dojo users in aggregate as a baseline.

\paragraph{Computer-aided proofreading.} We compare against focused proofreading by Plaza~\cite{focused_proofreading}. Focused proofreading performs graph analysis on the output from NeuroProof~\cite{neuroproof2013}, instead of our GALA approach. Therefore, for training our focused proofreading baseline, we replace GALA in our automatic segmentation pipeline with NeuroProof but use exactly the same input data including membrane probabilities. We obtained the best possible parameters for NeuroProof by consulting the developers (Tab.~\ref{tab:parameters}). Rather than using Raveler as the frontend, we use our own interface (Fig.~\ref{fig:ui}) to compare only the classifier from Plaza's approach.

\subsection{Experiments}

\paragraph{Selection oracle evaluation.} We use the selection oracle as described in Sec.~\ref{sec:errorcorrection} for the decision whether to accept or reject a correction. The purpose of this experiment is to investigate how many corrections are required to reach the best possible outcome. This is a direct comparison of the guided proofreading and focused proofreading classifiers but can only be performed if ground truth data is available. We perform this experiment on all datasets listed above.

\paragraph{Automatic method evaluation.} For this experiment, we accept all suggested corrections if the rankings are above a configured threshold $p_t=.95$ (Sec.~\ref{sec:errorcorrection}). We observed this value as stable in previous experiments with the guided proofreading classifiers (see supplementary material). We compare against the focused proofreading classifier and perform this experiment on all reported datasets.

\paragraph{Forced choice user experiments.} We conducted a quantitative user study to evaluate the forced choice setting (Sec.~\ref{sec:errorcorrection}). In particular, we evaluated how participants perform while correcting an automatic segmentation using the guided proofreading and focused proofreading tools. We designed a single factor between-subjects experiment with the factor \textit{proofreading classifier}, and asked participants to proofread the AC4 subvolume in a fixed time frame of 30 minutes. To enable comparison against the interactive proofreading study by~Haehn~\etal~\cite{haehn_dojo_2014}, we use the exact same study conditions, dataset, and time limit. The experiment was performed on a machine with standard off-the-shelf hardware. All participants received monetary compensation.

\paragraph{Novice study design.} We recruited participants with no experience in electron microscopy data or proofreading through flyers, mailing lists, and personal interaction. Based on sample size calculation theory, we estimated the study needed ten users per proofreading tool including four potential dropouts~\cite{samplesize1,samplesize2}. All twenty participants completed the study ($N=20$, 10 female; 19-65 years old, $M$=30). 

Each study session began with a five minute standardized explanation of the task. Then, the participants were asked to perform a 3 minute proofreading task on separate but representative data using focused proofreading. The participants were allowed to ask questions during this time. The classifier did not matter in this case since the user interface was the same. The experimenter then loaded the AC4 subvolume with initial pre-computed classifications by either guided proofreading or focused proofreading depending on assignment. After 30 minutes, the participants completed the raw NASA-TLX standard questions for task evaluation~\cite{NASATLX}.
\begin{figure*}[t]
\centering
\includegraphics[width=\linewidth]{ac4trails_combined.pdf}
\caption{Performance comparison of Plaza's focused proofreading (red) and our guided proofreading (blue) on the AC4 subvolume. All measurements are reported as median VI, the lower the better. We compare different approaches of accepting or rejecting corrections for each method: automatic selection with threshold (green line), forced choice by ten novice users, forced choice by two domain experts, and the selection oracle. In all cases, guided proofreading yields better results with fewer corrections.}
\label{fig:ac4trails}
\end{figure*}

\paragraph{Expert study design.} We recruited 4 domain experts to evaluate the performance of both guided and focused proofreading. We obtained study consent and randomly assigned 2 experts to proofread using each classifier. The experts performed the 3 minute test run on different data prior to proofreading for 30 minutes. After the task ended, the experts were asked to complete the raw NASA-TLX questionnaire.

\paragraph{Evaluation metric.} We measure the similarity between proofread segmentations and the manual `ground truth' labelings using \textit{variation of information} (VI). VI is a measure of the distance between two clusterings, closely related to mutual information (the lower, the better).


\section{Results and Discussion}

%We measure the performance of proofreading quantitatively by comparing VI scores of segmentations against ground truth labelings. Lower VI scores indicate less distance to the ground truth and a better segmentation. For all experiments, we report the VI score of the initial segmentation followed by the VI score of the proofreading output.
Additional plots are available as supplemental material due to reasons of space. 

\subsection{Precision and Recall}

\paragraph{L.~Cylinder.} Evaluation was performed on previously unseen sections of the mouse cortex volume from Kasthuri~\etal~\cite{kasthuri2015saturated}. We generated an unbalanced dataset of 81,184 correct and 8,780 split error patches with respect to the ground truth labeling. Then, we ranked each patch by using focused proofreading and guided proofreading, and compare precision/recall (Table \ref{tab:prcyl}). Our method exhibits higher precision and recall for correct and erroneous patches.

\begin{table}[t]
\caption{Classifier comparison on an unbalanced test set of the L.~Cylinder volume.}%While the training of our classifier is more expensive, testing accuracy is superior. }
\resizebox{\linewidth}{!}{
\begin{tabular}{lrrrr}
\toprule
 & Precision & Recall & F1 score & Test \# \\ 
\midrule
\emph{Focused Proofreading} & ~ & ~ & ~ & ~ \\ 
Correct & 0.93 & 0.31 & 0.47 & 81,184 \\ 
Split error & 0.11 & 0.78 & 0.19 & 8,780 \\ 
\emph{Guided Proofreading} & ~ & ~ & ~ & ~ \\ 
Correct & 1.00 & 0.93 & 0.96 & 81,184 \\ 
Split error & 0.61 & 0.96 & 0.74 & 8,780 \\ 
\bottomrule
\end{tabular} 
}
\label{tab:prcyl}
\end{table}

\paragraph{AC4 subvolume.} We generated 3,488 correct and 332 error patches (10 merge errors, 322 split errors). Guided proofreading achieves better precision/recall scores (Table \ref{tab:prac4}).

\begin{table}[t]
\caption{Classifier comparison on correct and split error patches of the AC4 subvolume.}%While the training of our classifier is more expensive, testing accuracy is superior. }
\resizebox{\linewidth}{!}{
\begin{tabular}{lrrrr}
\toprule
& Precision & Recall & F1 score & Test \# \\ 
\midrule
\emph{Focused Proofreading} & ~ & ~ & ~ & ~ \\ 
Correct & 0.94 & 0.69 & 0.80 & 3,488 \\ 
Split error & 0.14 & 0.51 & 0.21 & 332 \\ 
\emph{Guided Proofreading} & ~ & ~ & ~ & ~ \\ 
Correct & 1.00 & 0.92 & 0.96 & 3,488 \\ 
Split error & 0.54 & 0.95 & 0.69 & 332 \\ 
\bottomrule
\end{tabular} 
}
\label{tab:prac4}
\end{table}

\begin{figure}[t]
\centering
\includegraphics[width=\linewidth]{gfx/ac4boxplot.pdf}
\caption{VI distributions of guided proofreading (GP), focused proofreading (FP) and Dojo output across slices of the AC4 subvolume, with different error correction approaches. The variation resulting from performance of FP with automatic selection is $4.5\times$ higher than GP (as indicated by the arrow), with median VI of $1.9$ and $SD=0.496$.}
\label{fig:ac4boxplot}
\end{figure}

\subsection{Forced Choice User Experiment}
We performed a user study to evaluate the forced choice error correction method among novices and experts. To be comparable to Haehn~\etal's Dojo user study~\cite{haehn_dojo_2014}, participants were asked to proofread the AC4 subvolume for 30 minutes. We counted 10 merge errors and 322 split errors by computing the maximum overlap of the initial segmentation with respect to the ground truth labeling (provided in \cite{haehn_dojo_2014}). For evaluation, we measure the performance of proofreading quantitatively by comparing VI scores of segmentations. Median VI $=0.476$ ($SD=0.089$), with mean VI $=0.512$ ($SD=0.09$). Most novices and all experts were able to improve upon this score with both focused proofreading and guided proofreading (Fig.~\ref{fig:ac4trails}).

\paragraph{Novice performance.} Participants using focused proofreading were able to reduce the median VI of the automatic segmentation to $0.469$ ($SD=0.87$). On average, users viewed $423.4$ corrections and accepted $45.8$, with an average time of $4.9$ seconds per correction. Participants using guided proofreading were able to reduce the median VI to $0.424$ ($SD=0.037$). Here, users viewed on average $353.4$ corrections and accepted $106.9$, with an average correction time of $6.2$ seconds. While three users of focused proofreading made the initial segmentation worse, all participants using guided proofreading were able to improve it. In comparison to the results of Haehn~\etal, focused and guided proofreading outperform interactive proofreading with Dojo (median VI $0.535$, $SD=0.055$). The slope of VI score per correction (Fig.~\ref{fig:ac4trails}) shows that guided proofreading enables improvements with fewer corrections than the other tools. Interestingly, novice performance decreases after approximately $300$ corrections. There are two explanations for this: user fatigue, and increasing uncertainty during error suggestion from the classifier. %Regarding fatigue, we suggest that future experiments include short breaks after every ten minutes.

\paragraph{Expert performance.} Domain experts were able to improve the initial segmentation in all cases. With focused proofreading, the median VI of the automatic segmentation was $0.439$ ($SD=0.084$). With guided proofreading, the median VI was $0.396$ ($SD=0.032$, Fig.~\ref{fig:ac4boxplot}).

\paragraph{Subjective responses.} We used the NASA-TLX workload index to record subjective responses. Mental, physical, and temporal demands were reported slightly higher for participants using focused proofreading. However, these differences were not statistically significant. This is unsurprising as the user interface was the same for both groups.


\subsection{Automatic Error Correction}

\paragraph{Selection oracle.} As expected, the selection oracle yields the best performance on all datasets. Fig.~\ref{fig:ac4trails} shows VI reduction using the selection oracle on the AC4 subvolume (initial median VI $0.476$, $SD=0.089$). With focused proofreading, the selection oracle reaches a median VI of $0.353$ ($SD=0.037$) after $1600$ corrections. With guided proofreading, the oracle reaches a minimum median VI of $0.342$ ($SD=0.03$) after $800$ corrections. Both results are close to the best possible median VI of $0.334$ (calculated by computing maximum overlap with the ground truth). The slope of the trails in Fig.~\ref{fig:ac4trails} shows that guided proofreading requires fewer corrections to reach a reasonable reduction in VI. Fig.~\ref{fig:ac4boxplot} shows the VI distribution across methods. On the L.~Cylinder dataset (initial VI $0.379$, $SD=0.118$), focused proofreading reduces the median VI to $0.298$ ($SD=0.075$) after $26,170$ corrections ($2,419$ accepted). Guided proofreading reaches the minimum median VI $0.2996$ ($SD=0.073$) after $10,000$ corrections (in total $27,491$, $2,696$ accepted).

\paragraph{Automatic selection with threshold.} Focused proofreading was not designed to run automatically. This explains the poor performance on the AC4 subvolume (VI of $1.9$, $SD=0.496$) and on the L.~Cylinder dataset (VI of $2.75$, $SD=0.789$). For guided proofreading, we set $p_t=0.95$ for both datasets. This reduces median VI in the AC4 subvolume to $0.398$ ($SD=0.068$). This result is comparable to expert performance. Guided proofreading also reduces VI in the L.~Cylinder data to $0.352$ ($SD=0.087$).

\paragraph{Merge Error Detection.} Guided proofreading performs merge error detection prior to split error detection. The classifier found 10 merge errors in the AC4 subvolume, of which 4 reduced VI. Automatic selection with $p_t=0.95$ corrected 6 of these errors (Prec./Recall 0.87/0.80, F1-score 0.80). This was not captured in median VI, but resulted in a mean VI reduction from $0.512$ ($SD=0.09$) to $0.509$ ($SD=0.086$). The selection oracle reduced mean VI with only merge errors to $0.508$ ($SD=0.086$). In the forced choice user study, novices marked 1.9 merge errors for correction and reduced mean VI to $0.502$ (experts marked 2, VI $0.503$, $SD=0.086$). This shows how hard it is to identify merge errors. In 50 sections of the L.~Cylinder dataset, 151 merge errors were automatically found of which 17 reduced VI. Automatic selection with $p_t=0.95$ corrected 6 true VI-reducing errors and 30 VI-increasing ones (Prec./Recall 0.82/0.73, F1-score 0.77) to negligible VI effect. 

%\subsection{Limitations}
%Guided proofreading works on 2D image sections. This enables error correction without a computationally expensive alignment process. However, the output requires an additional (block-)merging step prior to 3D  analysis.


\section{Conclusions}

%The task of automatic cell boundary segmentation is difficult, and trying to improve such segmentations automatically as a post-process through merge and split error correction is, in principle, no different than trying to improve the underlying cell boundary segmentation. Due to the task difficulty, manual proofreading of connectomics segmentations is necessary, but it is a time consuming and error-prone task. Humans are the bottleneck and minimizing the manual labor is the goal.
%We have addressed this problem through training a convolutional neural network to detect ambiguous regions from labeled data---in effect, by finding a non-linear mapping between image and segmentation data. This allows us to identify merge and split errors with better performance than existing systems. Our experiments have shown that guided proofreading has the potential to reduce the bottleneck in the analysis of large connectomics datasets. To encourage testing of our proposed architecture and replicate our experiments, we provide our framework and data as free and open research at (link omitted for review).


{\small
\bibliographystyle{ieee}
\bibliography{connectomics}
}

\end{document}
