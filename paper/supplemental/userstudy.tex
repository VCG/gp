

\section{Forced Choice User Experiment}

\subsection{Recruitment and Participation}

Novice participants were recruited via flyer (figure~\ref{fig:flyer}). An anonymized listing of all participants including demographic information is shown in table~\ref{tab:participants}.

\begin{figure}[t]
\centering
\includegraphics[width=\linewidth]{gfx/flyer_anon.pdf}
\caption{Participants were recruited with this flyer.}
\label{fig:flyer}
\end{figure}

\begin{figure}[t]
\centering
\includegraphics[width=\linewidth]{gfx/NASA_TXL.pdf}
\caption{The NASA-TLX workload index to record subjective responses.}
\label{fig:nasatxl}
\end{figure}


\begin{table}[t]
\caption{The novice participants ($N=20$) of the forced choice user experiment. The table shows sex (20 female), age ($M=30$) and the randomly assigned classifier (focused proofreading as FP, guided proofreading as GP).}%While the training of our classifier is more expensive, testing accuracy is superior. }

\small{
\begin{tabular}{@{}l|c|c|c@{}}
	\toprule
     \textbf{ID} & \textbf{Sex} &  \textbf{Age} & \textbf{Classifier}  \\ \midrule	
S38 &		F & 20 & FP \\
S57&		F & 30 & FP \\ 
S32&		M & 38 & FP \\
S34&		F & 21 & FP \\
S21&		F & 65 & FP \\
S9 &	M & 33 & FP \\
S45 &		M & 28 & FP \\
S31&		M & 27 & FP \\
S24&		F & 21 & FP \\
S6	&	F & 38 & FP \\
S28	&	M&	32& GP \\
S36	&	F&	19& GP \\
S35	&	M&	26& GP \\
S25	&	M&	26& GP \\
S54	&	F&	30& GP \\
S53	&	M&	29& GP \\
S52	&	M	&27& GP \\
S51&		M&	31& GP \\
S200	 &	F	&37& GP \\
S3	 &F&	30 & GP



\end{tabular}
\hspace{2mm}
}
\label{tab:participants}
\end{table}

\subsection{User Interface}

\CHANGED{We integrate guided proofreading into an existing large data connectomics workflow. The web-based system is designed with a novice-friendly user interface (Fig. 5 in the paper, and the supplemental video). We show the current labeling of a cell boundary outline and its proposed correction overlayed on EM image data. The user cannot distinguish the current labeling from the proposed correction to avoid selection bias. We also show a solid overlay of the current and the proposed labeling. In addition, we show the image without overlays to provide an unoccluded view. User interaction is simple and involves one mouse click on either the current labeling or the correction. After interaction, the next potential error is shown.}

\subsection{Example Classifications}

During the user study, participants were asked to accept or reject potential errors and their corrections --- some more difficult than others. Figure~\ref{fig:patches} shows a selection of potential errors and their corrections.

\begin{figure}[t]
\centering
\includegraphics[width=\linewidth]{gfx/patches.pdf}
\caption{A selection of suggested errors and potential corrections during the forced choice user experiment. The star (*) indicates which choice reduces VI. While all participants were able to correctly choose for patch A, only few were able to correctly choose for patch B and C.}
\label{fig:patches}
\end{figure}

\subsection{Subjective Responses}

After the experiment, we acquired subjective responses using the NASA-TLX task load index (Figure~\ref{fig:nasatxl}). We performed ANOVA to test for statistical significance~\cite{shaffer1995}. Mental, physical, and temporal demands were reported slightly higher for participants using focused proofreading but the analysis did not yield any significance. \CHANGED{This is unsurprising as the user interface was the same for both groups.}

\begin{itemize}
\item \textbf{Mental Demand.} Participants using focused proofreading stated a higher mental demand $M=11.5$ ($SD=2.098$) than with guided proofreading $M=8.1$ ($SD=2.003$). This was not statistically significant ($F_{1,18}=3.2574, p=0.3695$).
\item \textbf{Physical Demand.} While naturally physical demand was rated low, participants using focused proofreading stated it slightly higher $M=5.4$ ($SD=2.26$) than with guided proofreading $M=2.9$ ($SD=1.76$). This was not statistically significant ($F_{1,18}=1.7507, p=0.5454$).
\item \textbf{Temporal Demand.} For temporal demand, participants using focused proofreading $M=8.4$ ($SD=1.95$) reported almost equal to guided proofreading $M=8.3$ ($SD=1.99$). This was not statistically significant ($F_{1,18}=0.0033, p=0.9987$).
\item \textbf{Performance.} Here, participants were asked to rate their own performance. All participants rated their performance as pretty well (the lower, the better). For focused proofreading $M=6.8$ ($SD=1.97$) and for guided proofreading $M=7.8$ ($SD=2.04$). This was not statistically significant ($F_{1,18}=0.3091, p=0.8878$).
\item \textbf{Effort.} Participants using focused proofreading stated higher effort $M=13.0$ ($SD=2.336$) than with guided proofreading $M=10.6$ ($SD=2.127$). This was not statistically significant ($F_{1,18}=1.1459, p=0.6599$).
\item \textbf{Frustration.} Participants overall reported low frustration. Reported were $M=5.0$ ($SD=1.90$) using focused proofreading and $M=5.9$ ($SD=185$) using guided proofreading. This was not statistically significant ($F_{1,18}=0.3271, p=0.8818$).
\end{itemize}