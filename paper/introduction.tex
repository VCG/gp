\section{Introduction}

In connectomics, neuroscientists annotate neurons and their connectivity within 3D volumes to gain insight into the functional structure of the brain. Rapid progress in automatic sample preparation and electron microscopy (EM) acquisition techniques has made it possible to image large volumes of brain tissue at $\approx4\, nm$ per pixel to identify cells, synapses, and vesicles. For $40\, nm$ thick sections, a $1\, mm^3$ volume of brain contains $10^{15}$ voxels, or 1 petabyte of data. With so much data, manual annotation is infeasible, and automatic annotation methods are needed~\cite{jain2010,Liu2014,GALA2014,kaynig2015large}.

%Automatic annotation by segmentation and classification of brain tissue is challenging~\cite{isbi_challenge}. The state of the art uses supervised learning with convolutional neural networks~\cite{Ciresan:2012f}, or potentially even unsupervised learning~\cite{BogovicHJ13}. Typically, cell membranes are detected in 2D images, and the resulting region segmentation is grouped into geometrically-consistent cells across registered sections. Cells may also be segmented across registered sections in 3D directly. Using dynamic programming techniques~\cite{Masci:2013a} and a GPU cluster, these classifiers can segment $\approx1$ terabyte of data per hour ~\cite{kasthuri2015saturated}. This is sufficient to keep up with the 2D data capture process on state-of-the-art electron microscopes (though 3D registration is still an expensive offline operation).

Automatic annotation by segmentation and classification of brain tissue is challenging~\cite{isbi_challenge} and all the methods make errors. This means we are left with large data which needs \emph{proofreading} by humans. This crucial task serves two purposes: 1) to correct errors in the segmentation, and 2) to provide a large body of labeled data to train better automatic segmentation methods. Recent proofreading tools provide intuitive user interfaces to browse segmentation data in 2D and 3D and to identify and manually correct errors ~\cite{markus_proofreading,raveler,mojo2,haehn_dojo_2014}. Many kinds of errors exist, such as inaccurate boundaries, but the most common are \emph{split errors}, where a single segment is labeled as two, and \emph{merge errors}, where two segments are labeled as one (Fig.~\ref{fig:merge_and_slit_errors}). With user interaction, split errors can be joined, and the missing boundary in a merge error can be defined with manually-seeded watersheds~\cite{haehn_dojo_2014}. However, even with semi-automatic correction tools, the visual inspection to find errors takes the majority of the time~\cite{proofreading_bottleneck}.

\begin{figure}[t]
\begin{center}
  \includegraphics[width=\linewidth]{gfx/merge_and_split_errors.png}
\end{center}
\vspace{-4mm}
   \caption{The most common proofreading corrections are fixing split errors (red arrows) and merge errors (yellow arrow). A fixed segmentation matches the cell borders.}
\label{fig:merge_and_slit_errors}
\end{figure}

Our goal is to add automatic detection of split and merge errors to proofreading tools. We design automatic classifiers that detect split and merge errors in segmentations so the user does not need to visually inspect the whole data volume to spot errors. A proofreading tool then recommends regions with a high probability of an error to the user, and suggest corrections to accept or reject. We call this process \textit{guided proofreading}.

In this paper, we introduce classifiers to detect merge- and split errors based on a convolutional neural network (CNN). We believe that this is the first time that deep learning is applied to the task of proofreading. Our classifiers work on top of any existing automatic segmentation method to find potential errors and suggest corrections. Given a membrane segmentation from a fast automatic method, our classifiers operate on the boundaries of whole cell regions. Compared to techniques that must analyze every input pixel, we reduce the data analysis to the boundaries. First, we train a CNN to detect only split errors. The output of this network is a probability whether a boundary between two segments is valid or not. We then reuse the same network to also detect merge errors by generating possible boundaries within a cell and inverting the split error score. We create corrections for both types of errors which can be accepted or rejected. This reduces the proofreading operation to simple yes/no decisions.

We further propose a greedy algorithm to perform proofreading. Possible erroneous regions are sorted by their score and the algorithm iteratively suggests a correction for each region. A user then works through this stream of regions and corrections. In a forced choice setting, the user either selects a correction or skips it to advance to the next region. This choice can be also performed automatically by running the algorithm until a configurable threshold is reached. In addition, if ground truth data is available, we can use a selection oracle to drive the forced choice selection. The oracle only accepts corrections which improve the automatic segmentation. This equals perfect proofreading.

We evaluate our method automatically by threshold and oracle on multiple real-world connectomics datasets. To evaluate the forced choice setting, we perform a quantitative user study. The study targets non-experts with no previous experience of proofreading electron microscopy data. We ask the participants to proofread a small segmentation volume in a fixed time frame by performing yes/no decisions. The user study is designed as a between-subjects experiment and compares guided proofreading against two other methods: a recently published fully interactive proofreading tool named \textit{Dojo} by Haehn~\etal~\cite{haehn_dojo_2014} and the semi-automatic \textit{focused proofreading} approach by Plaza~\cite{focused_proofreading}. We also asked four domain experts to use guided proofreading and focused proofreading for additional comparison.

Our first contribution is a classifier for split error detection based on a convolutional neural network. The classifier performs well even when trained with little amounts of training data. This is important since generating ground truth labels in connectomics requires manually labeling pixels and is very time-consuming. Our second contribution is a mechanism to identify merge-errors by re-using the split error classifier. Merge errors are usually less common than split errors in the oversegmented automatic labelings. However, they require more interaction during correction since split lines need to be manually drawn. Our method reduces this to a single click by providing the potential correction. The split and merge error identification is executed as a greedy algorithm to correct segmentation volumes, the third contribution of this paper. The algorithm can be driven automatically with a threshold, by an oracle based on ground truth and interactively in a forced choice setting. Our final contribution is our quantitative user study. We present statistically significant results showing that novice and expert users of guided proofreading are able to proofread a given dataset better and faster than with existing interactive and semi-automatic proofreading tools. As a consequence, we are able to provide tools to proofread segmentations more efficiently, and so better tackle large volumes of connectomics imagery.



%The reason why we work on top of an existing segmentation rather than improving the underlying technique, is to explore the error space in a non-linear way. The real world analogy would be inspecting something with a second set of eyes and different expertise.
